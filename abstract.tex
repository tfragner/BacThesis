%%%% Time-stamp: <2013-02-25 10:31:01 vk>


\chapter*{Abstract}
\label{cha:abstract}

Prozessmodellierung im Bereich des Business Process Management bedingt heutzutage, dass Prozessmodelle digital und ausführbar sind. Diese Arbeit stellt einen Algorithmus vor, mit dem die sequenzielle Abfolge von Aufgaben in kartengelegten Prozessen nach der Card Based Modeling Methode ermittelt und digitalisiert werden kann. Das Ergebnis bildet die Basis für eine spätere Umwandlung in ein ausführbares Modell. Hierzu werden Erkennungskriterien identifiziert. Diese Kriterien werden zur Formulierung einer Erkennungsstrategie genutzt, und ein davon abgeleiteter Algorithmus beschrieben. Die Arbeit zeigt anhand von Testbeispielen, dass der vorgestellte Algorithmus sehr gute Ergebnisse bei der Erkennung von sequenziell gelegten Prozessmodellen liefert, und als Basis für weitere Entwicklungen geeignet ist.  




%\glsresetall %% all glossary entries should be used in long form (again)
%% vim:foldmethod=expr
%% vim:fde=getline(v\:lnum)=~'^%%%%\ .\\+'?'>1'\:'='
%%% Local Variables:
%%% mode: latex
%%% mode: auto-fill
%%% mode: flyspell
%%% eval: (ispell-change-dictionary "en_US")
%%% TeX-master: "main"
%%% End:
