%----------------------------------------------------------------
%
%  File    :  einleitung.tex
%
%  Author  :  Thomas Fragner
% 
%  Created :  2017-10-01
% 
%  Changed :  2018-04-10
% 
%----------------------------------------------------------------

\chapter{Einleitung}
\label{cha:einleitung}
Business Process Management (BPM) ist seit Jahren das Interessengebiet vieler wissenschaftlicher Arbeiten in einem weiten Spektrum von Disziplinen, sowie ein treibender Faktor in der Industrie \cite{vanderAalst2016}. Ein wichtiger Bestandteil von BPM ist die Analyse und Modellierung von Prozessen, beginnend mit der Einführungsphase von BPM in Unternehmen. Die Verständlichkeit von Prozessmodellen ist zu jedem Zeitpunkt ein wichtiger Aspekt, und wird vom persönlichen Prozessmodellierungswissen der beteiligten Personen und der Komplexität der Prozessmodelle beeinflusst \cite{reijers_study_2011}. \citet{MENDLING2010127} beschreiben allgemeine Regeln zur Verständlichkeit von Modellen, die unabhängig von der gewählten Notation gültig sind, wie z.B. die Größe der Modelle oder die Struktur der Verzweigungen. Ein weiterer Aspekt der die Verständlichkeit beeinflusst ist der Abstraktionsgrad der Modelle. Nach \citet{vanderAalst2016} soll bei der Modellierung von Prozessen der Abstraktionsgrad des Modells den Anforderungen entsprechen, und der Fokus auf den Prozessen und nicht auf der Erstellung exakter und detaillierter Modell liegen, da z.B. beim Überarbeiten von Prozessen anfänglich ein abstraktes Modell ausreichend ist. Dies ist potenziell auch für die Erstanalyse von Prozessen gültig. 

\citet{Oppl:2015:ASB:2723839.2723841} beschreibt ein System zur Modellierung und Dokumentation von Prozessen, in dem vor allem Wert auf die aktive Partizipation der beteiligten Personen Wert gelegt wird, und schlägt hierzu eine Technik mit einer einfachen auf Kartenlegung basierenden Modellierungsmethode vor, welche zur Prozessmodellierung nur Subjekte (ausführende Personen/Rollen), Aufgaben, Austauschelemente (Nachrichten) und Verbindungselemente (Pfeile) verwendet, sowie einen Workflow zur Transformation in ein ausführbares S-BPM Modell. Folgend wird diese Methode als Card Based Modeling (CBM) bezeichnet. Aufgrund der geringen Anzahl von Notationselementen eignet sich diese Methode für die Darstellung von Prozessen mit hohem Abstraktionsgrad. Die Notwendigkeit der Transformation in ein automatisiert verarbeitbares S-BPM Modell wurde bereits durch einen Algorithmus realisiert mit dem Ergebnis, dass es noch Potential zur Verbesserung bei der Prozesserkennung gibt \cite{max}.

Die Notationselemente von CBM beschränken sich auf Elemente zur Darstellung von Aufgaben und deren Abhängigkeiten unter Berücksichtigung der ausführenden Subjekte. Nach \citet{Muehlen2013} sind Aufgaben, Verbindungen und Personen/Rollen die gebräuchlichsten Elemente in der Prozessmodellierung. Die Abhängigkeiten von Aufgaben werden in CBM durch die sequenzielle Anordnung der Aufgaben dargestellt. Sequenzielle Anordnungen sind auch in vielen anderen Modellierungssprachen wie z.B. BPMN zu finden.

Für die Transformation von CBM Modellen in digitale Repräsentationen ist daher die Erkennung von sequenziellen Kartenanordnungen ein wichtiger Bestandteil. Daraus leitet sich die Notwendigkeit eines Algorithmus zur möglichst fehlerfreien Erkennung eben dieser Anordnungen im Zuge der Digitalisierung ab.

\section{Forschungsaufgabe} % (fold)
\label{sec:forschungsaufgabe}
Aus der Notwendigkeit eines Algorithmus zur Erkennung von sequenziellen Abläufen in gelegten Prozessmodellen ergeben sich die folgenden Forschungsaufgaben:

\begin{enumerate}
	\item Um die Relevanz der Schaffung eines Algorithmus zu untermauern wird gezeigt, dass CBM und ähnliche Modellierungsmethoden nach den Regeln von \citet{MENDLING2010127} einfach verständliche Modellierungssprachen sind.
	\item Im Zuge der Digitalisierung von gelegten Prozessmodellen ist die Erkennung von sequenziellen Abläufen essenziell, da weitere Abhängigkeiten nur auf Basis einer fehlerfreien Erkennung dieser Strukturen richtig erkannt werden können. Ziel dieser Arbeit ist somit die Entwicklung eines allgemeinen Algorithmus zur Erkennung von sequenziell angeordneten Karten in gelegten Prozessmodellen anhand der CBM Methode \todo{beide Papers zitieren?} von \citet{oppl2016linking} unter Berücksichtigung bestehender Erkenntnisse \cite{max}, welche auch für andere derart gelagerte Problemstellungen in seinen Grundzügen verwendet werden kann.
\end{enumerate}
% section forschungsaufgabe (end)

\section{Methodik} % (fold)
\label{sec:methodik}
Zur Bearbeitung der ersten Forschungsaufgabe wird CBM nach den Regeln von \citet{MENDLING2010127} analysiert.

Für die Bearbeitung der zweiten Forschungsaufgabe werden die nachstehenden Schritte durchgeführt:
\begin{enumerate}
	\item Analyse der Kriterien für die Erkennung von sequenziell gelegten Prozessabfolgen.
	\item Entwicklung einer Erkennungsstrategie anhand der gefunden Kriterien.
	\item Erstellung eines abstrakten Algorithmus.
	\item Verifizierung des Algorithmus anhand von bestehenden Testbeispielen \cite{max}.
\end{enumerate}
% section methodik (end)

% chapter einleitung (end)

