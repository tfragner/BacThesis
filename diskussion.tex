%----------------------------------------------------------------
%
%  File    :  thesis-style.tex
%
%  Author  :  Keith Andrews, IICM, TU Graz, Austria
% 
%  Created :  27 May 93
% 
%  Changed :  19 Feb 2004
% 
% styling and technical implementation adopted 2011 by Karl Voit
%----------------------------------------------------------------

%% defined an anvironment for the style Keith used to use:

\chapter{Diskussion und Future Work}
\label{chap:Diskussion}
In Kapitel \ref{cha:Empfehlungen} wurde die Relevanz von CBM bezüglich der Verständlichkeit nach \citet{MENDLING2010127} gezeigt. Kapitel \ref{cha:hintergrund} zeigt die grundsätzliche Funktionsweise von CBM. Basierend auf CBM wurden in Kapitel \ref{cha:erkennung} die Kriterien für die Interpretation von gelegten Prozessmodellen bestimmt, welche in Kapitel \ref{sec:erkennungsstrategie} zur Formulierung einer Erkennungsstrategie verwendet wurden. Aus diesen Erkenntnissen wurde in Kapitel \ref{cap:algorithmus} der Algorithmus entwickelt, welcher in Kapitel \ref{chap:Ergebnisse} anhand von Testfällen verifiziert wurde. 

Die Ergebnisse zeigen, dass der vorgestellte Algorithmus die Anforderungen erfüllt und eine verbesserte Zuverlässigkeit im Vergleich zu \citet{max} feststellbar ist. Alle Testfälle wurden den Erwartungen entsprechend erkannt. Damit bildet der Algorithmus die Basis für weitere Schritte in der Digitalisierung von Prozessmodellen nach CBM. Zur vollständigen Überleitung in ein ausführbares S-BPM Modell müsste der Algorithmus um die Erkennung der Austauschelemente und deren Beziehungen erweitert werden. 